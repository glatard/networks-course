\documentclass{llncs}

\usepackage{amsmath} % for equation*
\usepackage{wasysym} % for \Box
\usepackage{color}
\usepackage{hyperref}
\usepackage{graphicx}
\definecolor{darkgreen}{rgb}{0,0.7,0}

\newcommand{\myspace}[0]{\vspace*{0.25cm}}

% Fix link colors
\hypersetup{
  colorlinks = true,
  linkcolor=red,
  citecolor=red,
  urlcolor=blue,
  linktocpage % so that page numbers are clickable in toc
}


\newcommand{\answer}[1]{}%{\color{red}\textit{#1}\color{black}}
\title{COMP 445 -- Theoretical Assignment 1 (TA1)\\ Winter 2018}

\author{Tristan Glatard\\
  \href{mailto:tristan.glatard@concordia.ca}{tristan.glatard@concordia.ca}
}

\institute{Concordia University\\
  Department of Computer Science and Software Engineering}

\begin{document}

\maketitle

\section*{Instructions}

\begin{itemize}
\item Please submit your assignment  as a pdf file on Moodle. The name of the pdf file must contain your name and student id. 
\item All questions will receive equal points.
\item Each question may have zero, one, or more than one
  correct choices.
\item Wrong answers will be penalized with negative
  points.
\item Partial answers will not
  receive any point.
\item Blank answers (no answer) will not be penalized.
\end{itemize}

\hrulefill\\

\myspace

\myspace

Student ID: \dotfill

\myspace

\myspace

First Name / Last Name: \dotfill

\myspace

\myspace

Signature: \dotfill

\myspace

\myspace

\hrulefill

\newpage

\section*{Introduction}

\paragraph{\textbf{Q1:}} A protocol of the application layer is implemented:\\

\begin{tabular}{ccl}
  a) & $\Box$ & At the network core. \\
  \\
  b) & $\Box$ & At the network edge. \\
  \\
  c) & $\Box$ & Both at the network core and at the network edge. 
\end{tabular}

\paragraph{\textbf{Q2:}} Consider two hosts A and B connected through a single router X (the
  network looks like A -- X -- B). Assume that the link between A and
  X is of capacity $R_{A-X}$=8~Gbps and the link between X and B is of
  capacity $R_{X-B}$=16~Gbps. What is the time required for N=5
  packets of size L=1~MB to be delivered from A to B, assuming that all
  delays except the transmission delay are negligible? We assume that 1~Gb=1000~Mb.\\
  
\begin{tabular}{ccl}
  a) & $\Box$ & 5.5~ms\\
  \\
  b) & $\Box$ & 7.5~ms\\
  \\
  c) & $\Box$ & 1.5~ms\\
  \\
  d) & $\Box$ & None of the above.
\end{tabular}

\paragraph{\textbf{Q3:}} Starting from the same network as in the previous question, we now assume that propagation delays are not negligible:
\begin{itemize}
\item A and X are connected by a 5000-km link ($l_{A-X}$=5000~km) where the propagation speed is s=$10^8$~m/s.
\item B and X are connected by a 20-km link ($l_{X-B}$=20~km) where the propagation speed is s=$10^8$~m/s.
\end{itemize}
What is the new delivery time between A and B for the same N=5 packets?\\

\begin{tabular}{ccl}
  a) & $\Box$ & 45.7~ms\\
  \\
  b) & $\Box$ & 55.5~ms \\
  \\
  c) & $\Box$ & 51.5~ms\\
  \\
  d) & $\Box$ & 55.7~ms
\end{tabular}

\paragraph{\textbf{Q4:}} In the Internet protocol stack, the transport layer can directly use services from:\\

\begin{tabular}{ccl}
  a) & $\Box$ & The application layer.\\
  \\
  b) & $\Box$ & The network layer.\\
  \\
  c) & $\Box$ & The link layer.\\
  \\
  d) & $\Box$ & All of the above.
\end{tabular}

\paragraph{\textbf{Q5:}} What is the probability that more than 5 users are active at the same time in a network of 15 users where each user is active 20\% of the time?\\

\begin{tabular}{ccl}
  a) & $\Box$ & 1\\
  \\
  b) & $\Box$ & 0.6\\
  \\
  c) & $\Box$ & 0.04\\
  \\
  d) & $\Box$ & 0.06
\end{tabular}

\section*{Application layer}

\paragraph{\textbf{Q6:}}
The content below was captured using
Wireshark:\\
\includegraphics[width=\textwidth]{dns.png}
This trace contains:\\

\begin{tabular}{ccl}
  a) & $\Box$ & An HTTP request encapsulated in a DNS query.\\
  \\
  b) & $\Box$ & A DNS query encapsulated in a TCP segment.\\
  \\
  c) & $\Box$ & An IP datagram encapsulated in a UDP datagram.\\
  \\
  d) & $\Box$ & A DNS query encapsulated in a UDP datagram.
\end{tabular}

\paragraph{\textbf{Q7:}} Among the following HTTP methods, which one(s) may be used to upload a file to a
Web server?\\

\begin{tabular}{ccl}
  a) & $\Box$ & GET\\
  \\
  b) & $\Box$ & POST\\
  \\
  c) & $\Box$ & PUT\\
  \\
  d) & $\Box$ & HEAD\\
\end{tabular}

\paragraph{\textbf{Q8:}} Among the following protocols, which one(s) are \textbf{not} involved in the retrieval of the Web page at URL \texttt{http://www.concordia.ca/} with a Web browser?\\

\begin{tabular}{ccl}
  a) & $\Box$ & TCP\\
  \\
  b) & $\Box$ & DNS\\
  \\
  c) & $\Box$ & HTTP\\
  \\
  d) & $\Box$ & SMTP\\
\end{tabular}

\paragraph{\textbf{Q9:}} DNS may be used to retrieve the name of the email server of a specific domain:\\

\begin{tabular}{ccl}
  a) & $\Box$ & by querying records of type CNAME.\\
  \\
  b) & $\Box$ & by querying records of type NS.\\
  \\
  c) & $\Box$ & through any type of iterated query.\\
  \\
  d) & $\Box$ & through any type of recursive query.\\
\end{tabular}

\paragraph{\textbf{Q10:}} SMTP is:\\

\begin{tabular}{ccl}
  a) & $\Box$ & a push protocol.\\
  \\
  b) & $\Box$ & a deprecated protocol.\\
  \\
  c) & $\Box$ & a transport protocol (a protocol belonging to the transport layer).\\
  \\
  d) & $\Box$ & a text (ASCII) protocol.\\
\end{tabular}

\end{document}
